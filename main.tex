\let\negmedspace\undefined{}
\let\negthickspace\undefined{}
\documentclass[journal,12pt,twocolumn]{IEEEtran}
 \usepackage{gensymb}
 \usepackage{polynom}
\usepackage{amssymb}
\usepackage[cmex10]{amsmath}
\usepackage{amsthm}
 \usepackage{stfloats}
\usepackage{bm}
 \usepackage{longtable}
 \usepackage{enumitem}
 \usepackage{mathtools}
 \usepackage{tikz}
 \usepackage[breaklinks=true]{hyperref}
\usepackage{listings}
\usepackage{color}                                            
\usepackage{array}                                            
\usepackage{longtable}                                        
\usepackage{calc}                                             
    \usepackage{multirow}                                         
    \usepackage{hhline}                                           
    \usepackage{ifthen}                                           
    \usepackage{lscape}     
\DeclareMathOperator*{\Res}{Res}
\DeclareMathOperator*{\equals}{=}
\renewcommand\thesection{\arabic{section}}
\renewcommand\thesubsection{\thesection.\arabic{subsection}}
\renewcommand\thesubsubsection{\thesubsection.\arabic{subsubsection}}
\renewcommand\thesectiondis{\arabic{section}}
\renewcommand\thesubsectiondis{\thesectiondis.\arabic{subsection}}
\renewcommand\thesubsubsectiondis{\thesubsectiondis.\arabic{subsubsection}}
\hyphenation{op-tical net-works semi-conduc-tor}
\def\inputGnumericTable{}                                 %%
\lstset{
frame=single, 
breaklines=true,
columns=fullflexible
}
\begin{document}
\newtheorem{theorem}{Theorem}[section]
\newtheorem{problem}{Problem}
\newtheorem{proposition}{Proposition}[section]
\newtheorem{lemma}{Lemma}[section]
\newtheorem{corollary}[theorem]{Corollary}
\newtheorem{example}{Example}[section]
\newtheorem{definition}[problem]{Definition}
\newcommand{\BEQA}{\begin{eqnarray}}
\newcommand{\EEQA}{\end{eqnarray}}
\newcommand{\define}{\stackrel{\triangle}{=}}
\newcommand*\circled[1]{\tikz[baseline= (char.base)]{
    \node[shape=circle,draw,inner sep=2pt] (char) {#1};}}
\bibliographystyle{IEEEtran}
\providecommand{\mbf}{\mathbf}
\providecommand{\pr}[1]{\ensuremath{\Pr\left(#1\right)}}
\providecommand{\qfunc}[1]{\ensuremath{Q\left(#1\right)}}
\providecommand{\sbrak}[1]{\ensuremath{{}\left[#1\right]}}
\providecommand{\lsbrak}[1]{\ensuremath{{}\left[#1\right.]}}
\providecommand{\rsbrak}[1]{\ensuremath{{}\left[#1\right.]}}
\providecommand{\brak}[1]{\ensuremath{\left(#1\right)}}
\providecommand{\lbrak}[1]{\ensuremath{\left(#1\right.)}
\providecommand{\rbrak}[1]{\ensuremath{\left[#1\right.]}}}
\providecommand{\cbrak}[1]{\ensuremath{\left\{#1\right\}}}
\providecommand{\lcbrak}[1]{\ensuremath{\left\{#1\right.}}
\providecommand{\rcbrak}[1]{\ensuremath{\left.#1\right\}}}
\theoremstyle{remark}
\newtheorem{rem}{Remark}
\newcommand{\sgn}{\mathop{\mathrm{sgn}}}
\providecommand{\abs}[1]{\left\vert#1\right\vert}
\providecommand{\res}[1]{\Res\displaylimits_{#1}} 
\providecommand{\norm}[1]{\left\lVert#1\right\rVert}
\providecommand{\mtx}[1]{\mathbf{#1}}
\providecommand{\mean}[1]{E\left[ #1 \right]}
\providecommand{\fourier}{\overset{\mathcal{F}}{ \rightleftharpoons}}
\providecommand{\system}{\overset{\mathcal{H}}{ \longleftrightarrow}}
\newcommand{\solution}{\noindent \textbf{Solution: }}
\newcommand{\cosec}{\,\text{cosec}\,}
\newcommand*{\permcomb}[4][0mu]{{{}^{#3}\mkern#1#2_{#4}}}
\newcommand*{\perm}[1][-3mu]{\permcomb[#1]{P}}
\newcommand*{\comb}[1][-1mu]{\permcomb[#1]{C}}
\renewcommand{\thetable}{\arabic{table}} 
\providecommand{\dec}[2]{\ensuremath{\overset{#1}{\underset{#2}{\gtrless}}}}
\newcommand{\myvec}[1]{\ensuremath{\begin{pmatrix}#1\end{pmatrix}}}
\newcommand{\mydet}[1]{\ensuremath{\begin{vmatrix}#1\end{vmatrix}}}
\numberwithin{equation}{section}
\numberwithin{figure}{section}
\numberwithin{table}{section}
\makeatletter
\@addtoreset{figure}{problem}
\makeatother
\let\StandardTheFigure\thefigure{}
\let\vec\mathbf{}
%\renewcommand{\thefigure}{\theproblem}
\def\putbox#1#2#3{\makebox[0in][l]{\makebox[#1][l]{}\raisebox{\baselineskip}[0in][0in]{\raisebox{#2}[0in][0in]{#3}}}}
     \def\rightbox#1{\makebox[0in][r]{#1}}
     \def\centbox#1{\makebox[0in]{#1}}
     \def\topbox#1{\raisebox{-\baselineskip}[0in][0in]{#1}}
     \def\midbox#1{\raisebox{-0.5\baselineskip}[0in][0in]{#1}}
\vspace{3cm}
\title{Assignment 3 9th Class Stats}
\author{Gunjit Mittal (AI21BTECH11011)}
\maketitle
Download all python codes from 
\begin{lstlisting}
https://github.com/GunjitMittal/Assignment3/tree/main/Assignment3/codes
\end{lstlisting}
Download all latex codes from 
\begin{lstlisting}
https://github.com/GunjitMittal/Assignment3/tree/main/Assignment3 
\end{lstlisting} 
\section{Question}
Three coins were tossed 30 times simultaneously. Each time the number of heads
occurring was noted down as follows:\\  
\begin{table}[ht!]
    \centering
    \input{tables/question.tex}\label{table:table1}	 
    \caption{}
    \label{Table 1}
\end{table}   \\
Prepare a frequency distribution table for the data given above.\\
\section{Solution}
\solution{}
Counting, we can see that in 30 throws we got 6 throws with no heads, 10 throws with 1 head, 9 throws with 2 heads and 5 throws with 3 heads
\begin{table}[ht!]
    \centering
    \input{tables/table1.tex}\label{table:table2}
    \caption{}
    \label{Table 2}	
\end{table} \\
Let the random variable $X \in \cbrak{0,1,2,3}$ denote the number of heads in the coin-tossing experiment. Now, 
\begin{equation}
   \pr{X = i} = \dfrac{n(X = i)}{\sum_{i=0}^{3} n(X = i) }
\end{equation}
where $i \in \cbrak{0,1,2,3}$ and n(X = i) is the frequency of getting i heads. Also,
\begin{align}
&\text{Number of times 3 coins were tossed} = 30\\ 
&\implies  \sum_{i=0}^{3} n(X = i) = 30
\end{align}
We have,
\begin{align}
\pr{X = 0} &= \frac{6}{30} = 0.20 \\
\pr{X = 1} &= \frac{10}{30} = 0.33 \\
\pr{X = 2} &= \frac{9}{30} = 0.30 \\
\pr{X = 3} &= \frac{5}{30} = 0.17
\end{align}
\begin{figure}[ht!]
     \centering
     \includegraphics[width = \columnwidth]{figs/PMF1.png}
     \caption{Plot of PMF using above data}
     \label{fig:Figure 1}
\end{figure}
\textbf{Now considering fair coins:}
Let probability of getting a head be a success and equal to p and probability of getting a tail be a failure and equal to q where $p+q = 1$. We can express this as a binomial distribution
\begin{equation}
   \sum_{i=0}^{n} \pr{X = i} =  \sum_{i=0}^{n} \comb{n}{i}(\text{p})^i\left(1-\text{p}\right)^{n-i}
\end{equation}
where $n = 3$ for 3 coins. Therefore,
\begin{equation}
\pr{X = i} = \comb{3}{i} (\text{p})^i (\text{q})^{3-i}    
\end{equation}
For fair coins, 
\begin{align}
   \text{p} &= \frac{1}{2}\\
  \therefore  \text{q} &= \frac{1}{2}
\end{align}
Therefore,
\begin{align}
    \pr{X = 0} &= \comb{3}{0}\brak{\frac{1}{2}}^0\brak{\frac{1}{2}}^3 = \frac{1}{8}\\
    \pr{X = 1} &= \comb{3}{1}\brak{\frac{1}{2}}^1\brak{\frac{1}{2}}^2 = \frac{3}{8}\\
    \pr{X = 2} &= \comb{3}{2}\brak{\frac{1}{2}}^2\brak{\frac{1}{2}}^1 = \frac{3}{8}\\
    \pr{X = 3} &= \comb{3}{3}\brak{\frac{1}{2}}^3\brak{\frac{1}{2}}^0 = \frac{1}{8}
\end{align} 
\begin{figure}[!ht]
    \centering
    \includegraphics[width = \columnwidth]{figs/PMF2.png}
    \caption{Comparison of theoretical and practical PMF plots}
\end{figure}
\end{document} 